% Options for packages loaded elsewhere
\PassOptionsToPackage{unicode}{hyperref}
\PassOptionsToPackage{hyphens}{url}
%
\documentclass[
  12pt,
]{article}
\usepackage{lmodern}
\usepackage{amsmath}
\usepackage{ifxetex,ifluatex}
\ifnum 0\ifxetex 1\fi\ifluatex 1\fi=0 % if pdftex
  \usepackage[T1]{fontenc}
  \usepackage[utf8]{inputenc}
  \usepackage{textcomp} % provide euro and other symbols
  \usepackage{amssymb}
\else % if luatex or xetex
  \usepackage{unicode-math}
  \defaultfontfeatures{Scale=MatchLowercase}
  \defaultfontfeatures[\rmfamily]{Ligatures=TeX,Scale=1}
\fi
% Use upquote if available, for straight quotes in verbatim environments
\IfFileExists{upquote.sty}{\usepackage{upquote}}{}
\IfFileExists{microtype.sty}{% use microtype if available
  \usepackage[]{microtype}
  \UseMicrotypeSet[protrusion]{basicmath} % disable protrusion for tt fonts
}{}
\makeatletter
\@ifundefined{KOMAClassName}{% if non-KOMA class
  \IfFileExists{parskip.sty}{%
    \usepackage{parskip}
  }{% else
    \setlength{\parindent}{0pt}
    \setlength{\parskip}{6pt plus 2pt minus 1pt}}
}{% if KOMA class
  \KOMAoptions{parskip=half}}
\makeatother
\usepackage{xcolor}
\IfFileExists{xurl.sty}{\usepackage{xurl}}{} % add URL line breaks if available
\IfFileExists{bookmark.sty}{\usepackage{bookmark}}{\usepackage{hyperref}}
\hypersetup{
  pdftitle={Black Lives Matter Protests and Voter Turnout},
  pdfauthor={Cameron Kimble; Leanne Fan; Kasey Zapatka; Kevin Morris},
  hidelinks,
  pdfcreator={LaTeX via pandoc}}
\urlstyle{same} % disable monospaced font for URLs
\usepackage[margin=1in]{geometry}
\usepackage{longtable,booktabs}
\usepackage{calc} % for calculating minipage widths
% Correct order of tables after \paragraph or \subparagraph
\usepackage{etoolbox}
\makeatletter
\patchcmd\longtable{\par}{\if@noskipsec\mbox{}\fi\par}{}{}
\makeatother
% Allow footnotes in longtable head/foot
\IfFileExists{footnotehyper.sty}{\usepackage{footnotehyper}}{\usepackage{footnote}}
\makesavenoteenv{longtable}
\usepackage{graphicx}
\makeatletter
\def\maxwidth{\ifdim\Gin@nat@width>\linewidth\linewidth\else\Gin@nat@width\fi}
\def\maxheight{\ifdim\Gin@nat@height>\textheight\textheight\else\Gin@nat@height\fi}
\makeatother
% Scale images if necessary, so that they will not overflow the page
% margins by default, and it is still possible to overwrite the defaults
% using explicit options in \includegraphics[width, height, ...]{}
\setkeys{Gin}{width=\maxwidth,height=\maxheight,keepaspectratio}
% Set default figure placement to htbp
\makeatletter
\def\fps@figure{htbp}
\makeatother
\setlength{\emergencystretch}{3em} % prevent overfull lines
\providecommand{\tightlist}{%
  \setlength{\itemsep}{0pt}\setlength{\parskip}{0pt}}
\setcounter{secnumdepth}{5}
\usepackage{rotating}
\usepackage{setspace}
\usepackage{booktabs}
\usepackage{longtable}
\usepackage{array}
\usepackage{multirow}
\usepackage{wrapfig}
\usepackage{float}
\usepackage{colortbl}
\usepackage{pdflscape}
\usepackage{tabu}
\usepackage{threeparttable}
\usepackage{threeparttablex}
\usepackage[normalem]{ulem}
\usepackage{makecell}
\usepackage{xcolor}
\ifluatex
  \usepackage{selnolig}  % disable illegal ligatures
\fi
\newlength{\cslhangindent}
\setlength{\cslhangindent}{1.5em}
\newlength{\csllabelwidth}
\setlength{\csllabelwidth}{3em}
\newenvironment{CSLReferences}[2] % #1 hanging-ident, #2 entry spacing
 {% don't indent paragraphs
  \setlength{\parindent}{0pt}
  % turn on hanging indent if param 1 is 1
  \ifodd #1 \everypar{\setlength{\hangindent}{\cslhangindent}}\ignorespaces\fi
  % set entry spacing
  \ifnum #2 > 0
  \setlength{\parskip}{#2\baselineskip}
  \fi
 }%
 {}
\usepackage{calc}
\newcommand{\CSLBlock}[1]{#1\hfill\break}
\newcommand{\CSLLeftMargin}[1]{\parbox[t]{\csllabelwidth}{#1}}
\newcommand{\CSLRightInline}[1]{\parbox[t]{\linewidth - \csllabelwidth}{#1}\break}
\newcommand{\CSLIndent}[1]{\hspace{\cslhangindent}#1}

\title{Black Lives Matter Protests and Voter Turnout\thanks{Prepared for the 2021 Annual Meeting of the American Sociological Association. Draft---please do not cite without permission.}}
\author{Cameron Kimble\footnote{Research and Program Associate, Brennan Center for Justice (\href{mailto:cameron.kimble@nyu.edu}{\nolinkurl{cameron.kimble@nyu.edu}})} \and Leanne Fan\footnote{PhD Student, Department of Sociology, CUNY Graduate Center (\href{mailto:lfan@gradcenter.cuny.edu}{\nolinkurl{lfan@gradcenter.cuny.edu}})} \and Kasey Zapatka\footnote{PhD Candidate, Department of Sociology, CUNY Graduate Center (\href{mailto:kzapatka@gradcenter.cuny.edu}{\nolinkurl{kzapatka@gradcenter.cuny.edu}})} \and Kevin Morris\footnote{PhD Student, Department of Sociology, CUNY Graduate Center (\href{mailto:kmorris@gradcenter.cuny.edu}{\nolinkurl{kmorris@gradcenter.cuny.edu}})}}
\date{July 15, 2021}

\begin{document}
\maketitle
\begin{abstract}
In the summer of 2020, Americans took to the streets in larger numbers than ever before demanding racial justice. Following the police murders of George Floyd and Breonna Taylor, an enormous multiracial coalition voiced their dissatisfaction with the state of policing in the United States. These mass protests took place shortly before a presidential election, and the incumbent loudly voiced his disdain for protesters and their political message. This paper explores one aspect of the impact of the Black Lives Matter (BLM) movement. Using a national voter file and data on protest location, we aim to estimate the causal impact of physical proximity to a BLM protest on voter turnout and Americans' political beliefs.
\end{abstract}

\pagenumbering{gobble}
\pagebreak
\doublespacing

\pagenumbering{arabic}

\emph{We are excited to join everyone virtually on August 10! Unfortunately, our coauthor with much of the data experienced a last-minute hard drive failure, resulting in the loss of some intermediate data files. As such, this draft is not as complete as we originally hoped. We will circulate a more updated version to the panel later in July. Thanks for your patience with us!}

In the United States, socioeconomically disadvantaged Black Americans are subject to disproportionate interactions with the state's repressive apparatuses -- the police, prisons, and the criminal justice system, more broadly -- contributing to what many consider a continuing devaluation of the lives of Black people. Emerging from this institutional backdrop Black Lives Matter began as a twitter hashtag following the acquittal of George Zimmerman in the 2013 fatal shooting of Trayvon Martin (\protect\hyperlink{ref-Rickford2016}{Rickford 2016}). In the wake of Officer Darren Wilson not being charged for the killing of unarmed 18-year-old Michael Brown, the BLM movement was catapulted to national prominence due to its visible role in the subsequent Ferguson protests (\protect\hyperlink{ref-Lowery2017}{Lowery 2017}). Since then, BLM has evolved into both an influential social justice movement composed of loosely affiliated local chapters nationwide, and the most widely recognized expression of Black resistance to racialized state violence.

The past decade has seen repeated outbreaks of BLM-led social protests against police killings of Black civilians, as well as the increasing ubiquity of the smartphone, and thus, cell phone video recordings of fatal incidents. Indeed, on May 25, 2020, Minneapolis Police Officer Derek Chauvin knelt on George Floyd Jr.'s neck for over 9 minutes, killing him. The bystander recording of the murder would ``go viral'' and provoke outrage in the United States and worldwide. Despite occurring amid a global pandemic, the scale and reach of the BLM protests that followed Mr.~Floyd's killing were unprecedented in U.S. history, with an estimated total of 7 million people participating in protests in nearly 550 localities across the country through early June (\protect\hyperlink{ref-Buchanan2020}{Buchanan, Bui, and Patel 2020}).
By Fall 2020, these movements have left an indelible mark on American popular, political, and corporate culture. This paper investigates the causal effect of geographical proximity to BLM protest on voter turnout and opinions on policing.

\hypertarget{protests-turnout}{%
\subsection*{Protests \& Turnout}\label{protests-turnout}}
\addcontentsline{toc}{subsection}{Protests \& Turnout}

A large body of work in sociology has examined the influence of social movements, large-scale protests, and electoral politics. Key to this literature has been understanding how individuals' beliefs and actions are shaped by social movements. \protect\hyperlink{ref-Lohmann1994}{Lohmann} (\protect\hyperlink{ref-Lohmann1994}{1994}) suggests that protest demonstrations reveal privately held information about dissatisfaction to the general public. Specifically, they raise awareness of social issues that result from prior policies with the aim of influencing voting decisions, as protest often signals intent to vote for or support a particular candidate or party. In going about their day-to-day lives, people acquire information regarding the consequences of various public policies, whether through factors such as changes in wealth, or general well-being, more broadly. At the level of individual experience, this may influence a single voter to prefer one candidate over another. But when people rally around an issue, through actions and demonstrations, this signals mass disaffection with the policy status quo. Collective action - a result of collective, shared experience - then serves as a more accurate reflection of the consequences of policies. This information, at one point dispersed among many individuals, is now made public, sending a ``loud'' signal to both other, less-informed voters as well as political elites.

Some researchers recognize the social environment as a mechanism to increase turnout by establishing norms of participation and providing networks to mobilize potential voters (\protect\hyperlink{ref-Rosenstone2003}{Rosenstone and Hansen 2003}; \protect\hyperlink{ref-Verba1995}{Verba, Schlozman, and Brady 1995}). Moreover, informal communication within social networks influences electoral behavior, as social pressure from within an individual's network works as an inducement to political participation (\protect\hyperlink{ref-Gerber2008}{Gerber, Green, and Larimer 2008}). In fact, simply being informed of another's actions in one's community -- or network -- can increase an individual's likelihood of voting (\protect\hyperlink{ref-Grosser2006}{Großer and Schram 2006}).

It is important to note, however, that much of this work has focused on the effect of social pressures on turnout within a local context. For instance, Grosser et al.~find that individuals, when informed of their neighbors' turning out to vote, are more likely to vote themselves. They conclude that this neighborhood information effect increases turnout by nearly 50 percent. Similarly, Gerber et al., observed substantially higher turnout rates among those who received pre-election mailings promising to publicize the fact that they voted for their neighbors. But, more than a decade later, with the rise of Facebook, Instagram, and other forms of web-based communications, individuals may choose to publicize their decision to vote to their friends and neighbors online, and likely to a wider audience.

While past work has focused on the relationship between geographical proximity and turnout, there is reason to believe that the BLM protests in 2020 operated differently than past social movements. The proliferation of social media, the nationalization of traditional media, and the polarization of Donald Trump's re-election campaign turned local protests into a national issue. The 2020 BLM protests demonstrate how protest movements can raise awareness of social discontent and increase its salience in the mind of the public. Informational and network effects were amplified with the advent of social media, and by facilitating information exchange, sites like Twitter and Facebook make it easier for large groups of people to organize collective demonstrations. However, unique to the modern era is how these demonstrations are portrayed. Wasow (\protect\hyperlink{ref-Wasow2020}{2020}) offers the concept of agenda seeding to describe how protestors drive media coverage and framing, congressional speech, and public opinion on civil rights, finding that nonviolent protests led to sympathy, especially when met with repression, and destructive protests led to themes centered on ``law and order.'' In the current political environment, portrayals of the protests differed along ideological lines, and so reinforced already existing political beliefs (see, for instance, \protect\hyperlink{ref-SignalAI2020}{\emph{Signal AI} 2020}).

Noting the gradual decline of local political news, researchers increasingly agree that American politics have become more and more nationalized in recent decades (\protect\hyperlink{ref-Fontana2018}{Fontana 2018}; \protect\hyperlink{ref-Moskowitz2021}{Moskowitz 2021}). Indeed, to say ``Donald Trump,'' or in this case, Black Lives Matter ``is to cue a set of meaningful associations with the national parties, the social groups that support them, and the positions that they take'' (\protect\hyperlink{ref-Hopkins2018}{Hopkins 2018}: 2). As such, we expect that proximity to a local BLM protest may have operated differently than social movements in the past. In particular, we expect the mobilizing effect at the local level was smaller than historical social movements. Given the polarization and prominence of national coverage, we suspect that individuals interpreted local protests through national media frames, and so were mobilized to vote for BLM and against Trump, or against ``rioting'' and for Trump. Whether the presence of social media and its heightening of network and informational effects motivated people to vote for or against Trump largely depends upon a person's preexisting political beliefs, and thus their preferred media. Put more simply, people motivated to vote by the BLM Movement voted in reaction to the national protest movement, not its local iterations.

Although we expect the local effects of BLM protests on \emph{who} voted to be relatively small, the protests may have influenced \emph{how} individuals voted. Scholars analyze the reciprocal relationship between social movements and elections more broadly than turnout alone. Political parties have attempted to capitalize on discontent by claiming issue ownership on social issues and including activists' perspectives in their party platforms (\protect\hyperlink{ref-Fetner2008}{Fetner 2008}). Grievance is harnessed by political parties to mobilize groups of voters(\protect\hyperlink{ref-Leege2002}{Leege et al. 2002}), and when social movements organized around a particular grievance challenge the status quo, mobilization efforts from the associated political party intensifies, in hopes of increasing voter turnout (\protect\hyperlink{ref-Winders1999}{Winders 1999}). As a result, protests that express liberal issues lead to a greater percentage of the two-party vote share for Democratic candidates, while protests that espouse conservative issues offer Republican candidates a greater share of the two-party vote (\protect\hyperlink{ref-Gillion2018}{Gillion and Soule 2018}). Put differently, there is little evidence of an electoral ``backlash''---that is, that social movements in support of liberal causes end up benefiting conservative candidates, and vice versa.

Demonstrating the reciprocal relationship between movements and elections, ``Vote and Organize'' became the motto for the BLM movement as election day grew nearer, the movement's ``get out the vote'' effort reaching over 60 million households, according to one prominent national BLM activist (\protect\hyperlink{ref-Cullors2020}{Cullors 2020}). At the same time, the Democratic Party continues to promote the BLM Movement and its issues.\footnote{See \url{https://democrats.org/black-lives-matter-organizing-resources/}} As such, while proximity to BLM protests likely did not affect voter turnout, the existence of the national BLM movement did.

\hypertarget{protest-political-beliefs}{%
\section*{Protest \& Political beliefs}\label{protest-political-beliefs}}
\addcontentsline{toc}{section}{Protest \& Political beliefs}

There is no consensus on whether protest movements help or hurt in advancing their purported goals -- affecting public opinion to change public policy. However, foundational research on political socialization indicates that initial frame alignment between an individual and a social movement organization is a precondition for participation (\protect\hyperlink{ref-Snow1986}{Snow et al. 1986}). That frame alignment is a continuous process that occurs between individuals and social movements, transforming and reinforcing the individual's ideological orientations. By their very nature, protest movements offer the opportunity for a frame realignment.

Research on how participation in (or exposure to) protest movements change individual political beliefs has largely focused on the Civil Rights Movement, a movement with which Black Lives Matter shares several similarities. One study found that participating in the Freedom Summer project ``radicalized'' volunteers, who were then significantly more likely to participate in social movements in subsequent years (\protect\hyperlink{ref-McAdam1989}{McAdam 1989}). Similarly, researchers conclude that involvement with social movements through the 1960s and 1970s caused a liberal shift in political orientation, as evidenced by subsequent political engagement such as voting for Jimmy Carter and participating in subsequent demonstrations (\protect\hyperlink{ref-Sherkat1997}{Sherkat and Blocker 1997}).

At a more macro level, Soumyajit Mazumder (\protect\hyperlink{ref-Mazumder2018}{2018}) finds that whites from counties that experienced civil rights protest are more liberal today, as measured by shifts in levels of racial resentment against African Americans, support for affirmative action, and identification with the Democratic Party. Moreover, this enduring attitudinal change is largely attributed to Civil Rights activists priming identities other than race (such as the American identity) and emphasizing ways in which whites and blacks were similar, causing whites to feel more sympathetic to black issues. On the contrary, Black Lives Matter activists invoke gender and LGBTQ+ frames in addition to race (\protect\hyperlink{ref-Tillery2019}{Tillery 2019}), which has been shown to reduce support for the movement, though not necessarily the movement's goals (\protect\hyperlink{ref-Bonilla2020}{Bonilla and Tillery 2020}).

Finally, \protect\hyperlink{ref-Madestam2013}{Madestam et al.} (\protect\hyperlink{ref-Madestam2013}{2013}) shows that Tea Party Movement rallies caused individual participants to become more politically conservative in their political views, increased public support for Tea Party positions, and led to more Republican votes in the 2010 midterm elections. Specifically, they find that the interactions produced at rallies and protests caused genuine shifts in political views, which sustained the momentum for a rightwards shift in fiscal policy going forward. We anticipate that the local Black Lives Matter protests had a similar effect among Democratic-leaning voters.

This is in line with the ``social logic of politics,'' (\protect\hyperlink{ref-Zuckerman2005}{Zuckerman 2005}) which centers social learning through interaction with others in development of political beliefs. Focusing events,'' unexpected and visible events that harm a specific sub-population, such as the cellphone video of George Floyd's killing, push issues related to said event to the top of public consciousness (\protect\hyperlink{ref-Birkland1998}{Birkland 1998}). Moreover, theories of activated opinion suggest that minority-led protests often serve as a ``bottom-up'' factor that powers liberal shifts in public opinion (\protect\hyperlink{ref-Lee2002}{Lee 2002}). Thus, we anticipate that BLM protests caused those already sympathetic to BLM issues -- Black voters and Democrats -- to view the police in a less positive light.

At the same time, the police are a well-known and widely trusted institution, specifically among White Americans (\protect\hyperlink{ref-PewResearchCenter2019}{Center 2019}). And given the centrality of police in popular imagination, as well as their receiving continued elite support, views on policing are difficult to change (\protect\hyperlink{ref-Zaller1992}{Zaller 1992}; \protect\hyperlink{ref-Nakamura2020}{Nakamura and Hermann 2020}). We expect that while some white voters, those that lean Democratic, will view police less positively as a result of the BLM movement, the majority of white voters identify as Republicans, and so will associate the BLM protests with chaos. Most Republican voters will therefore view police more positively because of the protests; though we expect a minority of Republican voters to report less positive views of police following BLM protests, so that overall, white voter opinions on police remains constant. In short, we expect that the BLM protests probably further polarized Americans on issues surrounding policing.

\hypertarget{data-and-design}{%
\section*{Data and Design}\label{data-and-design}}
\addcontentsline{toc}{section}{Data and Design}

We use a variety of data sources and empirical approaches to understand the electoral consequences of the 2020 protests in support of the Black Lives Matter movement. We are interested ultimately in the \emph{causal} effect of protest exposure on these beliefs and behaviors, complicating our empirical approach. It seems highly probable that factors associated with protest formation are also associated with other political behaviors. Put differently, demonstrating a correlation between protest exposure and voting behavior might simply point to a third factor influencing them both. An example might be helpful: it seems possible that areas with large Black populations were politicized by the Trump Administration's response to the COVID-19 pandemic, which disproportionately impacted Black communities. This politicization---which may have occurred prior to the murder of George Floyd---could have increased an area's propensity to protest \emph{and} increased their likelihood of voting in November. Thus, any correlation between protest and turnout would be due not to the protests themselves, but the underlying politicization.

To estimate the causal effect of protests on voter turnout, we leverage the known fact that protests are less likely to develop in inclement weather (e.g. \protect\hyperlink{ref-Madestam2013}{Madestam et al. 2013}). We use an instrumental variables (IV) approach that allows us to leverage random fluctuations in rainfall, the resulting protest formation, and eventual turnout and political behavior. IV setups are a common way for identifying causal relationships in the social sciences (\protect\hyperlink{ref-Morgan2007}{Morgan and Winship 2007}; \protect\hyperlink{ref-Bollen2012}{Bollen 2012}; \protect\hyperlink{ref-Sovey2011}{Sovey and Green 2011}).

IV models do, however, have one very strict assumption: namely, the \emph{exclusion restriction.} In other words, we must assume that our exogenous variable (here, rainfall) is unrelated to our dependent variable in any way other than through our endogenous variable (here, protest formation). If there is reason to believe that rainfall influenced turnout in November or political beliefs, our estimates cannot determine the causal relationship. There is reason to be particularly concerned with a violation of this assumption when using rainfall: as \protect\hyperlink{ref-Mellon2021}{Mellon} (\protect\hyperlink{ref-Mellon2021}{2021}) shows, nearly 200 journal articles have used rainfall as an instrument in recent years. That so many scholars have used rainfall as an instrument---that is, have argued that rainfall has a causal effect on many sorts of socially meaningful outcomes---challenges the validity of the assumption that rainfall impacts turnout only through protest formation. It is incumbent upon us, then, to demonstrate that the use of rainfall does not violate the exclusion restriction.

One way of improving the validity of our instrument is to use a region's relative rainfall, not overall rainfall. It is possible, for instance, that protests were higher in areas of the country with generally rainy weather. Portland, Oregon, for instance, saw hundreds of protests over the course of 2020 and is notoriously rainy. If we used only absolute rainfall, we might incorrectly assume that protest formation was larger in generally rainier areas. Moreover, an additional inch of rain in a place like Oregon might impact protest formation less than in an arid climate like Phoenix. George Floyd was murdered on May 25, and protests began in the days that followed, and peaked the weekend of June 6 (\protect\hyperlink{ref-Buchanan2020}{Buchanan, Bui, and Patel 2020}). As such, our instrumental variable is the amount of rainfall that fell between May 26 and June 7, 2020, scaled by historical rainfall. Specifically, we follow \protect\hyperlink{ref-Cooperman2017}{Cooperman} (\protect\hyperlink{ref-Cooperman2017}{2017}) in constructing a z-score for each weather station. The z-score is calculated as \(\frac{x_{it} - \bar{x_{i}}}{s_{i}}\), where \emph{x\textsubscript{it}} is weather station \emph{i}'s 2020 rainfall; \(\bar{x}\)\emph{\textsubscript{i}} is weather station \emph{i}'s mean rainfall between May 26 and June 7 over the 1948--2019 period; and \emph{s\textsubscript{i}} is the standard deviation of the annual rainfall of weather station \emph{i} over the 1948--2020 period.

It is still possible, however, that relative rainfall in late May and early June could influence our dependent variables through other avenues. This seems especially likely if places with high relative rainfall in May and June \emph{also} had high rainfall in late October---such high rainfall could then reduce turnout (\protect\hyperlink{ref-Hansford2010}{Hansford and Gomez 2010}). However, relative rainfall in the May--June period is only weakly associated with relative rainfall between October 28 and November 3 (\emph{r} = 0.04). We thus conclude that our approach satisfies the exclusion restriction. Put differently, we assume that rainfall in late June and early May impacted November turnout \emph{only} through protest formation.

\hypertarget{data}{%
\subsection*{Data}\label{data}}
\addcontentsline{toc}{subsection}{Data}

Our data come from a variety of sources. To estimate relative rainfall, we turn to the National Oceanic and Atmospheric Administration (NOAA), which collects detailed weather data from around the country. NOAA records the estimated daily rainfall at some 13,000 locations around the country. We use the \texttt{rnoaa} (\protect\hyperlink{ref-Chamberlain2021}{Chamberlain 2021}) package to download and process this data. The weather sites include each site's latitude and longitude.

Our data on protest formation comes from the U.S. Crisis Monitor, a project of the Armed Conflict Location \& Event Data Project (ACLED) and the Bridging Divides Initiative (BDI) at Princeton University.\footnote{See \url{https://acleddata.com/special-projects/us-crisis-monitor/}.} The U.S. Crisis Monitor compiled geocoded data on BLM protests from around the United States throughout 2020. Using this data, we identify over 3,800 Black Lives Matter protests occurring between May 26 and June 7. These protests occurred in all 50 states and Washington, DC.

\hypertarget{voter-file-data}{%
\subsection*{Voter File Data}\label{voter-file-data}}
\addcontentsline{toc}{subsection}{Voter File Data}

To explore the relationship between protest exposure and individual-level turnout, we leverage the national registered voter file provided by data vendor L2. L2 collects the registered voter file from each state in the country and includes a host of self-reported and modelled information such as age, partisan affiliation, gender, and race / ethnicity. Importantly, they also indicate whether each voter participated in the 2020 general election, as well as past contests. L2 geocodes these records and maps them to their home census block groups. Taken as a whole, we have the individual-level voter records for the more than 200 million registered voters in the United States. Each record includes the voter's historical turnout and their party of registration.\footnote{In states without party registration, L2 uses proprietary models to estimate it, largely based on the primaries that voters have participated in.}

There are some limitations to using the registered voter file to test for turnout effects. Most importantly, by looking at participation among registered voters, we may be ``selecting'' on the dependent variable (See \protect\hyperlink{ref-Nyhan2017}{Nyhan, Skovron, and Titiunik 2017}). In other words, registration \emph{itself} is a form of political participation. If protest organizers registered many new voters at the protests---an approach that did apparently occur (\protect\hyperlink{ref-Timm2020}{Timm 2020})---but only a relatively small share of these new registrants actually voted, the net result may have been \emph{lower} turnout among registered voters (because the numerator grew less quickly than the denominator) but \emph{higher} turnout among eligible citizens (additional ballots are introduced against a stable estimate of the eligible population unconditioned on registration).

To mitigate this potential problem, we aggregate individual records up to the census block group and tract level to calculate the number of ballots cast in each unit in 2020 and previous elections. We use Census Bureau data on the citizen voting-age population (CVAP) as a denominator for turnout that is not biased by differential registration rates. Turnout is thus calculated as the number of ballots cast according to the voter file, divided by the CVAP for that geographical unit (tract or block group). We also use the registered voter file to determine the share of each unit that is registered as a Democrat or Republican. These estimates are then merged with demographic data from the 2015-2019 ACS 5-year survey such as median income, educational attainment, and median age.

Using an inverse distance weighting approach, we estimate the rainfall z-score for each unit in the country. This means that a given tract or block group's 's z-score will strongly reflect the z-score of the nearest weather station, but will also incorporate information from other nearby weather stations. The distance to a protest for each unit is measured from the centroid of the unit to the closest BLM protest. If a protest occurred anywhere within the unit, this distance is set to 0.

\emph{H1}: We expect that geographical proximity to BLM protests caused higher{]} block group and tract turnout, all else being held equal.

\emph{H2}: We expect that, however, that there was racial heterogeneity in the causal effects of living near a BLM protest. Specifically we expect that, all else equal, less-Black neighborhoods were demobilized by BLM protests, while Black neighborhoods were mobilized by these protests.

\hypertarget{addressing-spatial-dependence}{%
\subsection*{Addressing Spatial Dependence}\label{addressing-spatial-dependence}}
\addcontentsline{toc}{subsection}{Addressing Spatial Dependence}

Our empirical strategy is inherently spatial by design: we are interested in the causal effects of physical distance to a protest, and our instrumental variable leverages geographical variation in rainfall. These spatial aspects demand that we pay special econometric attention to spatial relationships. \emph{Spatial dependence}, or \emph{spatial autocorrelation}, arises when ``values observed at one location or region\ldots{} depend on the values of neighboring observations at nearby locations'' (\protect\hyperlink{ref-LeSage2009}{LeSage and Pace 2009, 2}). Since nearer things are more alike than distant things, \emph{ceteris paribus}, a value on a given spatial unit is more alike its neighbors than those farther away (\protect\hyperlink{ref-Tobler1970}{Tobler 1970}). It is this similarity of values clustered in space that violates the ``regression assumption that the error terms of individual observations can be considered independent of one another'' (\protect\hyperlink{ref-Ward2008}{Ward and Gleditsch 2008}, pg 33). When researchers fail to adjust for spatial dependence, it results in a larger than warranted \(t\) value that increases the chance of Type 1 error and biases the estimates (\protect\hyperlink{ref-Ward2008}{Ward and Gleditsch 2008}).

There are three primary spatial relationships in our project. First, census block group turnout in one block group is likely to be very similar, and therefore spatially autocorrelated, with the turnout of those nearby because some characteristic that motivates political participation is likely clustered in space. Figure \ref{fig:maps} provides visual evidence of this spatial dependence (first-order spatial effects) as we see clusters of dark (low turnout) and light (high turnout) values. Second, we are also concerned by spatial dependence in the outcome variable in our first stage model---distance from a protest. The distance of one block group is closely related to those of its neighbors because that block group and its neighbors by definition and block group construction are similar distances from a protest. Figure \ref{fig:maps} also presents clear evidence of first-order spatial effects in that we see obvious clusters of both larger and smaller distances from BLM protests. Thirdly, we are also concerned about spatial dependence in our instrument, rainfall, since rainfall is not randomly distributed because neighboring block groups are more likely to ``experience similar weather on any given day, and therefore our instrument naturally exhibits spatial correlation'' (\protect\hyperlink{ref-Hansford2010}{Hansford and Gomez 2010, 281}; see also \protect\hyperlink{ref-Betz2020}{Betz, Cook, and Hollenbach 2020}). Figure \ref{fig:maps} illustrates this point very clearly.

\begin{figure}
\hypertarget{fig:maps}{%
\centering
\includegraphics{"../temp/maps.png"}
\caption{Spatial Distribution of Key Variables}\label{fig:maps}
}
\end{figure}

We confirm the statistical presence (second-order spatial effects) of spatial dependence with a global correlation coefficient known as a \emph{Moran's I} statistic, which tests for ``a linear association between a value and a weighted-average of its neighbors'' and is often used as a test for spatial autocorrelation (\protect\hyperlink{ref-Ward2008}{Ward and Gleditsch 2008, 23}; \protect\hyperlink{ref-Bivand2013}{Bivand, Pebesma, and Gómez-Rubio 2013, 261}). Table \ref{tab:moran} reports significant, \emph{preliminary}, Moran's I values for all three key variables: census block group turnout, distance from a protest, and rainfall across the country. Moran's I values for turnout suggest moderate, positive autocorrelation (0.37), whereas distance reports weak, positive autocorrelation (0.25). Moran's I for rainfall suggests very high, positive spatial autocorrelation (0.95). These results mean OLS and IV estimates are not reliable and we must correct for spatial dependence in both stages.

\begin{figure}
\hypertarget{tab:moran}{%
\centering
\includegraphics{"../temp/moran.png"}
\caption{Moran's I of Key Variables}\label{tab:moran}
}
\end{figure}

For both the first and second stages, we plan to run a SDEM instead of a SDM for three reasons.\footnote{We avoid running a SARAR (SAC) model because there are numerous drawbacks, specifically its sensitivity to misspecification (see \protect\hyperlink{ref-LeSage2014}{LeSage 2014})} First, we suspect local, not global effects. By this we mean that although ``neighbors affect each other, this effect does not propagate throughout the entire space'' (\protect\hyperlink{ref-Burkey2018}{Burkey 2018, 17}). We think neighboring block groups will be more similar to those close by but not to all block groups in the country or even in the same state. Second, we prefer SDEM to SDM because it allows for lagged values of predictors. In the first stage, we correct for the non-random distribution of our instrument, rainfall, which we suspect will have effects on our outcome by definition of IV design, but also local effects on contiguous block groups, not all block groups throughout the country. In the second stage, the share of black voters or population density (along with other controls) will have effects on our outcome and therefore their spatial dependence will need to be controlled for. Third, SDEM is more efficient in our case. Although SDEM does not explicitly model spatial dependence in the form of a lagged outcome \(\rho{Wy}\), which we certainly suspect, ``spatial dependence from the dependent variable will land in the disturbances'' \(\lambda{Wu}\), which are likely to produce a ``significant coefficient estimate for \(\lambda\) that might be similar in magnitude to the true value of \(\rho\)'' (\protect\hyperlink{ref-LeSage2014}{LeSage 2014, 14}). Our SDEM models will take the following form:

\(y = \alpha{} + X\beta_1{} + XW\beta_2{} + u,\) where \$ u = \lambda{Wu} + \epsilon{}\$

The Likelihood Ratio (LR) approach then allows us to test whether reduced models are more appropriate in diagnosing the specific forms of spatial dependence we expect.\footnote{We use a modern approach to determining model specification that is preferable to a classical approach (see \protect\hyperlink{ref-LeSage2009}{LeSage and Pace 2009}; \protect\hyperlink{ref-Burkey2018}{Burkey 2018} for a discussion of the relative merits.)} Using results from the LR tests and comparing the resulting ``residuals for remaining spatial association'' in our spatially adjusted models with those of reduced form (Spatial Error Models (SEM), Spatially Lagged X Model (SLX), and OLS) will help us determine which model specification, and therefore which estimates are most reliable in our IV setup (\protect\hyperlink{ref-Ward2008}{Ward and Gleditsch 2008, 28}). We plan to replicate this analysis for our ZIP-code-level data for similar reasons.

\hypertarget{survey-data}{%
\subsection*{Survey Data}\label{survey-data}}
\addcontentsline{toc}{subsection}{Survey Data}

While the administrative data gives us exceptional coverage of the relationships between protest and turnout across the country, turnout is a relatively blunt measure of political participation. Moreover, it seems possible that the BLM protests could have shifted voters' \emph{beliefs} about topics such as the police without bringing new participants into the voting booth.

To understand the effect of protest exposure on political beliefs we turn to national survey data. We use two national surveys: the American National Election Study 2020 Time Series data (ANES) and the Cooperative Election Study (CES). Both are widely used among sociologists and political scientists to understand the political orientation of the American electorate. In addition to a rich set of data about individuals' political beliefs, these surveys also collect information about respondents' age, race, family income, and other characteristics. Each record includes the respondent's home ZIP code. The current draft includes analysis of the CES data, while we intend to incorporate data from the ANES in a future draft. The 2020 CES includes data from 61,000 respondents, weighed to represent the national adult population

Once again, our empirical strategy relies on using rainfall as an instrument for protest exposure. In the case of the national survey data, however, there are not respondents from all ZIP codes in the country. The process for estimating the distance to protest using a rainfall z-score for each ZIP code in the country is identical to that described above for tracts and block groups.

\hypertarget{constructing-political-attitudes-in-the-cces}{%
\subsubsection*{Constructing Political Attitudes in the CCES}\label{constructing-political-attitudes-in-the-cces}}
\addcontentsline{toc}{subsubsection}{Constructing Political Attitudes in the CCES}

The addition of survey data to the previous analysis could illuminate potential theoretical pathways to understand how geographic proximity impacted. Using the 2020 Cooperative Election Study, we tease out the potential pathways through which the 2020 presidential voter turnout could be impacted by the protests for racial justice five months earlier. This analysis comes in two parts. We begin by re-estimating the effect of geographical proximity on turnout in the survey data, replicating the national models run using voter file data. Importantly, the survey data allows us to test a key causal mechanism hinted at in the literature: namely, the role played by local media. The specific questions used from the CES can be found in the Supplemental Information. The survey data allows us to test the following hypotheses:

\emph{H3}: Geographical proximity to a BLM protest led to higher turnout among respondents to the CES.

\emph{H4}: Geographical proximity to BLM protests had larger effects for respondents who reported more exposure to local news.

Further, we hypothesize that while the protests might have changed political attitudes---even if they did not impact turnout.. We use the survey to understand the extent to which the protests affect people's attitudes around racial resentment and criminal justice. In particular, we ask whether geographical proximity caused changes in racial beliefs among CES respondents, and whether it impacted respondents' views of American police.

Scholars commonly use the ``racial resentment'' score to analyze how racist survey respondents are (\protect\hyperlink{ref-Wilson2011}{Wilson and Davis 2011}). The racial resentment score is built from a battery of four questions, originally proposed by \protect\hyperlink{ref-Kinder1981}{Kinder and Sears} (\protect\hyperlink{ref-Kinder1981}{1981}) and included in ANES surveys since 1988. These questions aim to captures symbolic racism as ``a blend of antiblack affect and the kind of traditional American moral values embodied in the Protestant Ethic\ldots a form of resistance to change in the racial status quo based on moral feelings that blacks violate such traditional American values as individualism and self-reliance, the work ethic, obedience, and discipline'' (\protect\hyperlink{ref-Kinder1981}{Kinder and Sears 1981, 416}).\footnote{Despite the ubiquity of the racial resentment, this measure has been criticized for its imprecision---it is not entirely clear that these questions only captures racial resentment, specifically whether ``black Americans are undeserving of special considerations on the basis of racial group membership alone,'' as opposed to additionally capturing racial bigotry, stereotype or prejudice (\protect\hyperlink{ref-Wilson2011}{Wilson and Davis 2011, 119}).} In the CES 2020 survey, we develop a racial resentment index using 5 questions, 3 of which are only given to self-reported non-White respondents. The survey responses from these questions are averaged to create a single ``racial resentment index,'' regardless of respondent race, to a 1-5 scale, 5 being the most ``racially resentful.'' The questions used to construct the racial resentment scores are included in the Appendix.

Finally, in 2020 CES, there is a new suite of 8 criminal justice and policing questions. While this paper is among the first to leverage these questions in an academic study, they are conceived of as a ``police attitude index'' (see \protect\hyperlink{ref-Agadjanian2021}{Agadjanian 2021}). The 8 questions used to construct this index are included in the Appendix. The survey responses from these questions are averaged to create a single ``police attitude index,'' to a 1-2 scale, 2 being the most ``pro-police.''

These questions about racial resentment and attitudes toward the police allow us to test our final two hypotheses:

\emph{H5}: Geographical proximity to a BLM protest reduced respondents' racial resentment scores. This effect was particularly pronounced for respondents with large local news diets.

\emph{H6}: Geographical proximity to a BLM protest caused respondents to have worse views of the police, especially for respondents with large exposure to local news.

Future drafts of this paper will investigate whether there is evidence of ``backlash'' effects among conservatives. In particular, we expect that geographical proximity to BLM protests caused conservatives to have \emph{more favorable} attitudes toward the police.

\hypertarget{results}{%
\section*{Results}\label{results}}
\addcontentsline{toc}{section}{Results}

\hypertarget{overall-turnout-effects}{%
\subsection*{Overall Turnout Effects}\label{overall-turnout-effects}}
\addcontentsline{toc}{subsection}{Overall Turnout Effects}

In Table \ref{tab:bg-tract-dist} we demonstrate that, as expected, higher rainfall is strongly associated with increased distance to the nearest BLM protest even after controlling for other pertinent neighborhood characteristics that should influence protest formation such as age, race, and partisan affiliation. A one-standard deviation increase in a geographical unit's rainfall relative to the average is associated with an increase of roughly 3-4\% in distance to the nearest protest.

\textbf{INSERT TABLE ABOUT HERE. TEMPORARILY LOST DUE TO HARD DRIVE FAILURE.}

In Table \ref{tab:bg-tract-to} we present the results of the full instrumental variables models for block groups (models 1 and 2) and tracts (models 3 and 4). In models 1 and 3 we present the overall turnout effects of living near a protest, while models 2 and 4 ask whether the causal effect was moderated by the share of the tract or block group that is non-Hispanic Black.

In each of these models, distance to the nearest protest is instrumented by rainfall in the geographical unit, lending causal power to the interpretation of the relationship. However, as discussed above, this draft does not yet account for the spatial dependencies in rainfall; as such, these estimates should presently be taken with a healthy dose of skepticism. Despite this caveat, the remainder of the draft will interpret the coefficients with the assumption that they are causally credible.

In both models 1 and 3, we see that living near a BLM protest in the summer of 2020 caused lower turnout overall. Models 2 and 4, however, support our expectation that there would be racial heterogeneity in these estimates: while living near a BLM protest was apparently demobilizing for non-Black communities, models 2 and 4 indicate that living close to a protest was apparently far less demobilizing---or even mobilizing---for Black tracts and block groups.

\begin{figure}
\hypertarget{tab:bg-tract-to}{%
\centering
\includegraphics{"../temp/reg.png"}
\caption{Causal Effect of Distance to Protest on Turnout}\label{tab:bg-tract-to}
}
\end{figure}

In Figure \ref{fig:mef-t-to} we present the marginal effects plot from model 4 of Table \ref{tab:bg-tract-to}. We show the relationships between race, distance to protest, and tract turnout, with all other covariates held at their means. The figure indicates that the estimated effect sizes are very large, raising questions about the validity of the causal design. We expect these estimated effect sizes will shrink substantially once we control for spatial dependencies in the rainfall.

\begin{figure}
\hypertarget{fig:mef-t-to}{%
\centering
\includegraphics{"../temp/mef.png"}
\caption{Census Tract Turnout in 2020, by Share Black and Instrumented DIstance to a BLM Protest}\label{fig:mef-t-to}
}
\end{figure}

The mobilizing effects of the BLM protests, then, were complicated: although they may have mobilized some Black neighborhoods, they apparently led to decreased turnout in areas with fewer Black residents.

Of course, the fact that BLM protests structured turnout does not show us the causal mechanisms through which it did so. In the following section, we turn to national survey data to explore the causal effect of proximity to protest on such political beliefs as orientation toward the police and racial resentment, and also ask whether the geographical proximity effects on turnout were structured by the amount of news voters received from local---as opposed to national---sources.

\hypertarget{turnout-proximity-and-political-beliefs}{%
\subsection*{Turnout, Proximity, and Political Beliefs}\label{turnout-proximity-and-political-beliefs}}
\addcontentsline{toc}{subsection}{Turnout, Proximity, and Political Beliefs}

Having established that geographical proximity led to lower turnout for Americans in 2020---especially for non-Black Americans---we turn to national survey data to better understand these relationships. As discussed above, the national turnout models give us a tremendous amount of analytical power and geographical coverage---our models incorporate information from every registered voter in the country, giving us enormous breadth in our estimates. Nevertheless, the administrative data cannot explain how proximity to protests shaped Americans' beliefs, and the avenue through which proximity structured turnout. National survey data allows us to explore these relationships.

We begin by leveraging responses to the Cooperative Election Study (CES). From CES survey results, we employ the following dependent variables: voter turnout by political views, and police attitude index.

We chose these two dependent variables to further tease out the effects of geographic proximity, local/national news diet, and critically understand whether the impacts of the protest on racial attitude are more broadly around racial attitudes or more narrowly around criminal justice and police accountability.

In Table \ref{tab:ces-to} we re-estimate the effect of geographical proximity to a BLM protest on turnout. In model 2, we interact this effect with the respondent's news diet. Both models make clear that turnout was depressed by living near a protest, but that there were negligible effects for individuals who consumed both local and national news.

\begin{singlespace}
\input{"../temp/ces_to_clean.tex"}
\end{singlespace}

In Table \ref{tab:ces-views} we present the causal effects of geographical proximity to a BLM protest on respondents' racial resentment (in models 1 and 2) and their views of the police (in models 3 and 4). Smaller distances to protests were associated with lower racial resentment, indicating that protests may have reduced the racist views of Americans. Although this was true for all groups, it was especially pronounced for individuals with national news consumption. Living closer to a protest had no overall effect on police views, though there was heterogeneity in this treatment effect. Individuals who consumed only local news saw \emph{increased} police support in response to protests, while those who consumed any amount of national news lowered their opinions of the police in response to the protests.

\begin{singlespace}
\input{"../temp/ces_views_clean.tex"}
\end{singlespace}

\hypertarget{supplementary-information}{%
\section*{Supplementary Information}\label{supplementary-information}}
\addcontentsline{toc}{section}{Supplementary Information}

\hypertarget{questions-from-the-ces}{%
\subsection*{Questions from the CES}\label{questions-from-the-ces}}
\addcontentsline{toc}{subsection}{Questions from the CES}

The racial resentment scores in the CES are constructed using the following questions, where respondents rank their agreement with the statement on a scale from 1 to 5:

\begin{itemize}
\item
  Irish, Italians, Jewish and many other minorities overcame prejudice and worked their way up. Blacks should do the same without any special favors.
\item
  Generations of slavery and discrimination have created conditions that make it difficult for blacks to work their way out of the lower class.
\item
  {[}Only asked to non-White respondents{]} I resent when Whites deny the existence of racial discrimination.
\item
  {[}Only asked to non-White respondents{]} Whites get away with offenses that African Americans would never get away with.
\item
  {[}Only asked to non-White respondents{]} Whites do not go to great lengths to understand the problems African Americans face.
\end{itemize}

The police support index is constructed using the following questions, where respondents indicate whether they support or disagree with a policy.

\begin{itemize}
\item
  Eliminate mandatory minimum sentences for non-violent drug offenders.
\item
  Require police officers to wear body cameras that record all of their activities while on duty.
\item
  Increase the number of police on the street by 10 percent, even if it means fewer funds for other public
\item
  Decrease the number of police on the street by 10 percent, and increase funding for other public services
\item
  Ban the use of choke holds by police
\item
  Create a national registry of police who have been investigated for or disciplined for misconduct.
\item
  End the Department of Defense program that sends surplus military weapons and equipment to police
\item
  Allow individuals or their families to sue a police officer for damages if the officer is found to have ``recklessly disregarded'' the individual's rights.
\end{itemize}

\newpage

\hypertarget{references}{%
\section*{References}\label{references}}
\addcontentsline{toc}{section}{References}

\hypertarget{refs}{}
\begin{CSLReferences}{1}{0}
\leavevmode\hypertarget{ref-Agadjanian2021}{}%
Agadjanian, Alexander. 2021. {``Is the {Relationship} Between {Police Attitudes} and {Hispanics Shifting} to {Trump} in 2020 {Distinctively Strong}?''} {Agadjanian Politics}. March 28, 2021. \url{https://agadjanianpolitics.wordpress.com/2021/03/27/is-the-relationship-between-police-attitudes-and-hispanics-shifting-to-trump-in-2020-distinctively-strong/}.

\leavevmode\hypertarget{ref-Betz2020}{}%
Betz, Timm, Scott J. Cook, and Florian M. Hollenbach. 2020. {``Spatial Interdependence and Instrumental Variable Models.''} \emph{Political Science Research and Methods} 8 (4): 646--61. \url{https://doi.org/10.1017/psrm.2018.61}.

\leavevmode\hypertarget{ref-Birkland1998}{}%
Birkland, Thomas A. 1998. {``Focusing {Events}, {Mobilization}, and {Agenda Setting}.''} \emph{Journal of Public Policy} 18 (1): 53--74. \url{https://doi.org/10.1017/S0143814X98000038}.

\leavevmode\hypertarget{ref-Bivand2013}{}%
Bivand, Roger, Edzer J. Pebesma, and Virgilio Gómez-Rubio. 2013. \emph{Applied Spatial Data Analysis with {R}}. Second edition. Use {R}! {New York}: {Springer}.

\leavevmode\hypertarget{ref-Bollen2012}{}%
Bollen, Kenneth A. 2012. {``Instrumental {Variables} in {Sociology} and the {Social Sciences}.''} \emph{Annual Review of Sociology} 38: 37--72. \url{http://www.jstor.org/stable/23254586}.

\leavevmode\hypertarget{ref-Bonilla2020}{}%
Bonilla, Tabitha, and Alvin B. Tillery. 2020. {``Which {Identity Frames Boost Support} for and {Mobilization} in the \#{BlackLivesMatter Movement}? {An Experimental Test}.''} \emph{American Political Science Review} 114 (4): 947--62. \url{https://doi.org/10.1017/S0003055420000544}.

\leavevmode\hypertarget{ref-Buchanan2020}{}%
Buchanan, Larry, Quoctrung Bui, and Jugal K. Patel. 2020. {``Black {Lives Matter May Be} the {Largest Movement} in {U}.{S}. {History}.''} \emph{The New York Times: U.S.}, July 3, 2020. \url{https://www.nytimes.com/interactive/2020/07/03/us/george-floyd-protests-crowd-size.html}.

\leavevmode\hypertarget{ref-Burkey2018}{}%
Burkey, Mark L. 2018. {``Spatial {Econometrics} and {GIS YouTube Playlist}.''} \emph{REGION} 5 (3): R13--18. \url{https://doi.org/10.18335/region.v5i3.254}.

\leavevmode\hypertarget{ref-PewResearchCenter2019}{}%
Center, Pew Research. 2019. {``Why {Americans Don}'t {Fully Trust Many Who Hold Positions} of {Power} and {Responsibility}.''} {Pew Research Center}. \url{https://www.pewresearch.org/politics/2019/09/19/why-americans-dont-fully-trust-many-who-hold-positions-of-power-and-responsibility/}.

\leavevmode\hypertarget{ref-Chamberlain2021}{}%
Chamberlain, Scott. 2021. \emph{Rnoaa: '{NOAA}' {Weather Data} from {R}}. \url{https://CRAN.R-project.org/package=rnoaa}.

\leavevmode\hypertarget{ref-Cooperman2017}{}%
Cooperman, Alicia Dailey. 2017. {``Randomization {Inference} with {Rainfall Data}: {Using Historical Weather Patterns} for {Variance Estimation}.''} \emph{Political Analysis} 25 (3): 277--88. \url{https://doi.org/10.1017/pan.2017.17}.

\leavevmode\hypertarget{ref-Cullors2020}{}%
Cullors, Patrisse. Letter. 2020. {``Dear {President}-{Elect Joe Biden} and {Vice}-{President}-{Elect Kamala Harris},''} November 7, 2020. \url{https://blacklivesmatter.com/wp-content/uploads/2020/11/blm-letter-to-biden-harris-110720.pdf}.

\leavevmode\hypertarget{ref-Fetner2008}{}%
Fetner, Tina. 2008. \emph{How the Religious Right Shaped Lesbian and Gay Activism}. Social Movements, Protest, and Contention, v. 31. {Minneapolis}: {University of Minnesota Press}.

\leavevmode\hypertarget{ref-Fontana2018}{}%
Fontana, David. 2018. {``No, {All American Politics Isn}'t {National}.''} \emph{Bloomberg}, August 6, 2018. \url{https://www.bloomberg.com/news/articles/2018-08-06/is-all-american-politics-really-national-now}.

\leavevmode\hypertarget{ref-Gerber2008}{}%
Gerber, Alan S., Donald P. Green, and Christopher W. Larimer. 2008. {``Social {Pressure} and {Voter Turnout}: {Evidence} from a {Large}-{Scale Field Experiment}.''} \emph{American Political Science Review} 102 (1): 33--48. \url{https://doi.org/10.1017/S000305540808009X}.

\leavevmode\hypertarget{ref-Gillion2018}{}%
Gillion, Daniel Q., and Sarah A. Soule. 2018. {``The {Impact} of {Protest} on {Elections} in the {United States}*.''} \emph{Social Science Quarterly} 99 (5): 1649--64. \url{https://doi.org/10.1111/ssqu.12527}.

\leavevmode\hypertarget{ref-Grosser2006}{}%
Großer, Jens, and Arthur Schram. 2006. {``Neighborhood {Information Exchange} and {Voter Participation}: {An Experimental Study}.''} \emph{American Political Science Review} 100 (2): 235--48. \url{https://doi.org/10.1017/S0003055406062137}.

\leavevmode\hypertarget{ref-Hansford2010}{}%
Hansford, Thomas G., and Brad T. Gomez. 2010. {``Estimating the {Electoral Effects} of {Voter Turnout}.''} \emph{The American Political Science Review} 104 (2): 268--88. \url{http://www.jstor.org/stable/40863720}.

\leavevmode\hypertarget{ref-Hopkins2018}{}%
Hopkins, Daniel J. 2018. \emph{The Increasingly {United States}: How and Why {American} Political Behavior Nationalized}. Chicago Studies in {American} Politics. {Chicago}: {The University of Chicago Press}.

\leavevmode\hypertarget{ref-Kinder1981}{}%
Kinder, Donald R., and David O. Sears. 1981. {``Prejudice and Politics: {Symbolic} Racism Versus Racial Threats to the Good Life.''} \emph{Journal of Personality and Social Psychology} 40 (3): 414--31. \url{https://doi.org/10.1037/0022-3514.40.3.414}.

\leavevmode\hypertarget{ref-Lee2002}{}%
Lee, Taeku. 2002. \emph{Mobilizing Public Opinion: {Black} Insurgency and Racial Attitudes in the Civil Rights Era}. Studies in Communication, Media, and Public Opinion. {Chicago}: {University of Chicago Press}.

\leavevmode\hypertarget{ref-Leege2002}{}%
Leege, David C., Kenneth D. Wald, Brian S. Krueger, and Paul D. Mueller. 2002. \emph{The Politics of Cultural Differences: Social Change and Voter Mobilization Strategies in the Post-{New Deal} Period}. {Princeton}: {Princeton University Press}.

\leavevmode\hypertarget{ref-LeSage2014}{}%
LeSage, James P. 2014. {``What {Regional Scientists Need} to {Know} about {Spatial Econometrics}.''} \emph{Review of Regional Studies}, January. \url{https://doi.org/10.52324/001c.8081}.

\leavevmode\hypertarget{ref-LeSage2009}{}%
LeSage, James P., and Robert Kelley Pace. 2009. \emph{Introduction to {Spatial Econometrics}}. 0th ed. {Chapman and Hall/CRC}. \url{https://doi.org/10.1201/9781420064254}.

\leavevmode\hypertarget{ref-Lohmann1994}{}%
Lohmann, Susanne. 1994. {``Information {Aggregation Through Costly Political Action}.''} \emph{The American Economic Review} 84 (3): 518--30. \url{http://www.jstor.org/stable/2118065}.

\leavevmode\hypertarget{ref-Lowery2017}{}%
Lowery, Wesley. 2017. {``Black {Lives Matter}: Birth of a Movement.''} \emph{The Guardian: World News}, January 17, 2017. \url{http://www.theguardian.com/us-news/2017/jan/17/black-lives-matter-birth-of-a-movement}.

\leavevmode\hypertarget{ref-Madestam2013}{}%
Madestam, Andreas, Daniel Shoag, Stan Veuger, and David Yanagizawa-Drott. 2013. {``Do {Political Protests Matter}? {Evidence} from the {Tea Party Movement}.''} \emph{The Quarterly Journal of Economics} 128 (4): 1633--85. \url{https://doi.org/10.1093/qje/qjt021}.

\leavevmode\hypertarget{ref-Mazumder2018}{}%
Mazumder, Soumyajit. 2018. {``The {Persistent Effect} of {U}.{S}. {Civil Rights Protests} on {Political Attitudes}.''} \emph{American Journal of Political Science} 62 (4): 922--35. \url{https://doi.org/10.1111/ajps.12384}.

\leavevmode\hypertarget{ref-McAdam1989}{}%
McAdam, Doug. 1989. {``The {Biographical Consequences} of {Activism}.''} \emph{American Sociological Review} 54 (5): 744--60. \url{https://doi.org/10.2307/2117751}.

\leavevmode\hypertarget{ref-Mellon2021}{}%
Mellon, Jonathan. 2021. {``Rain, {Rain}, {Go Away}: 176 {Potential Exclusion}-{Restriction Violations} for {Studies Using Weather} as an {Instrumental Variable}.''} SSRN Scholarly Paper ID 3715610. {Rochester, NY}: {Social Science Research Network}. \url{https://doi.org/10.2139/ssrn.3715610}.

\leavevmode\hypertarget{ref-Morgan2007}{}%
Morgan, Stephen L., and Christopher Winship. 2007. \emph{Counterfactuals and Causal Inference: Methods and Principles for Social Research}. Analytical Methods for Social Research. {New York}: {Cambridge University Press}.

\leavevmode\hypertarget{ref-Moskowitz2021}{}%
Moskowitz, Daniel J. 2021. {``Local {News}, {Information}, and the {Nationalization} of {U}.{S}. {Elections}.''} \emph{American Political Science Review} 115 (1): 114--29. \url{https://doi.org/10.1017/S0003055420000829}.

\leavevmode\hypertarget{ref-Nakamura2020}{}%
Nakamura, David, and Peter Hermann. 2020. {``After Announcing Modest Police Reforms, {Trump} Pivots Quickly to a Law-and-Order Message in Appeal to His Base.''} \emph{Washington Post}, June 26, 2020. \url{https://www.washingtonpost.com/politics/after-announcing-modest-police-reforms-trump-pivots-quickly-to-a-law-and-order-message-in-appeal-to-his-base/2020/06/26/622c6688-b7b6-11ea-a8da-693df3d7674a_story.html}.

\leavevmode\hypertarget{ref-Nyhan2017}{}%
Nyhan, Brendan, Christopher Skovron, and Rocío Titiunik. 2017. {``Differential {Registration Bias} in {Voter File Data}: {A Sensitivity Analysis Approach}.''} \emph{American Journal of Political Science} 61 (3): 744--60. \url{https://doi.org/10.1111/ajps.12288}.

\leavevmode\hypertarget{ref-Rickford2016}{}%
Rickford, Russell. 2016. {``Black {Lives Matter}: {Toward} a {Modern Practice} of {Mass Struggle}.''} \emph{New Labor Forum} 25 (1): 34--42. \url{https://doi.org/10.1177/1095796015620171}.

\leavevmode\hypertarget{ref-Rosenstone2003}{}%
Rosenstone, Steven J, and John Mark Hansen. 2003. \emph{Mobilization, Participation, and Democracy in {America}}. {New York}: {Longman}.

\leavevmode\hypertarget{ref-Sherkat1997}{}%
Sherkat, Darren E., and T. Jean Blocker. 1997. {``Explaining the {Political} and {Personal Consequences} of {Protest}.''} \emph{Social Forces} 75 (3): 1049--70. \url{https://doi.org/10.2307/2580530}.

\leavevmode\hypertarget{ref-SignalAI2020}{}%
\emph{Signal AI}. 2020. {``Media Bias in the Coverage of {George Floyd},''} June 19, 2020. \url{https://www.signal-ai.com/blog/media-bias-in-the-coverage-of-george-floyd}.

\leavevmode\hypertarget{ref-Snow1986}{}%
Snow, David A., E. Burke Rochford, Steven K. Worden, and Robert D. Benford. 1986. {``Frame {Alignment Processes}, {Micromobilization}, and {Movement Participation}.''} \emph{American Sociological Review} 51 (4): 464--81. \url{https://doi.org/10.2307/2095581}.

\leavevmode\hypertarget{ref-Sovey2011}{}%
Sovey, Allison J., and Donald P. Green. 2011. {``Instrumental {Variables Estimation} in {Political Science}: {A Readers}' {Guide}.''} \emph{American Journal of Political Science} 55 (1): 188--200. \url{https://doi.org/10.1111/j.1540-5907.2010.00477.x}.

\leavevmode\hypertarget{ref-Tillery2019}{}%
Tillery, Alvin B. 2019. {``What {Kind} of {Movement} Is {Black Lives Matter}? {The View} from {Twitter}.''} \emph{Journal of Race, Ethnicity, and Politics} 4 (2): 297--323. \url{https://doi.org/10.1017/rep.2019.17}.

\leavevmode\hypertarget{ref-Timm2020}{}%
Timm, Jane C. 2020. {``Voter Registration Surged During {BLM} Protests, Study Finds.''} \emph{NBC News}, August 11, 2020. \url{https://www.nbcnews.com/politics/elections/voter-registration-surged-during-blm-protests-study-finds-n1236331}.

\leavevmode\hypertarget{ref-Tobler1970}{}%
Tobler, W. R. 1970. {``A {Computer Movie Simulating Urban Growth} in the {Detroit Region}.''} \emph{Economic Geography} 46 (June): 234. \url{https://doi.org/10.2307/143141}.

\leavevmode\hypertarget{ref-Verba1995}{}%
Verba, Sidney, Kay Lehman Schlozman, and Henry E. Brady. 1995. \emph{Voice and Equality: Civic Voluntarism in {American} Politics}. {Cambridge, Mass}: {Harvard University Press}.

\leavevmode\hypertarget{ref-Ward2008}{}%
Ward, Michael Don, and Kristian Skrede Gleditsch. 2008. \emph{Spatial Regression Models}. Quantitative Applications in the Social Sciences 155. {Thousand Oaks}: {Sage Publications}.

\leavevmode\hypertarget{ref-Wasow2020}{}%
Wasow, Omar. 2020. {``Agenda {Seeding}: {How} 1960s {Black Protests Moved Elites}, {Public Opinion} and {Voting}.''} \emph{American Political Science Review} 114 (3): 638--59. \url{https://doi.org/10.1017/S000305542000009X}.

\leavevmode\hypertarget{ref-Wilson2011}{}%
Wilson, David C., and Darren W. Davis. 2011. {``Reexamining {Racial Resentment}: {Conceptualization} and {Content}.''} \emph{The ANNALS of the American Academy of Political and Social Science} 634 (1): 117--33. \url{https://doi.org/10.1177/0002716210388477}.

\leavevmode\hypertarget{ref-Winders1999}{}%
Winders, Bill. 1999. {``The {Roller Coaster} of {Class Conflict}: {Class Segments}, {Mass Mobilization}, and {Voter Turnout} in the {U}.{S}., 1840-1996.''} \emph{Social Forces} 77 (3): 833--62. \url{https://doi.org/10.2307/3005963}.

\leavevmode\hypertarget{ref-Zaller1992}{}%
Zaller, John. 1992. \emph{The Nature and Origins of Mass Opinion}. \url{https://doi.org/10.1017/CBO9780511818691}.

\leavevmode\hypertarget{ref-Zuckerman2005}{}%
Zuckerman, Alan S., ed. 2005. \emph{The Social Logic of Politics: Personal Networks as Contexts for Political Behavior}. {Philadelphia}: {Temple University Press}.

\end{CSLReferences}

\end{document}
